\documentclass[12pt]{article}
\usepackage[spanish]{babel}
\usepackage{amsmath, amssymb}
\usepackage{circuitikz}
\usepackage{graphicx}
\usepackage{booktabs}
\usepackage{geometry}
\geometry{a4paper, margin=2cm}

\title{Análisis Detallado de Circuitos RL y RC en Serie}
\author{Tu Nombre}
\date{\today}

\begin{document}

\maketitle

% ===== SECCIÓN 1: INTRODUCCIÓN =====
\section{Introducción}
Este documento presenta el análisis completo de circuitos RL y RC en serie, incluyendo:
\begin{itemize}
    \item Resolución de ecuaciones diferenciales (EDOs)
    \item Aplicación de Transformada de Laplace
    \item Obtención de funciones de transferencia
    \item Comparación de respuestas temporales
\end{itemize}

% ===== SECCIÓN 2: CIRCUITO RL =====
\section{Circuito RL en Serie}

\subsection{Modelado Matemático}
\begin{circuitikz}[scale=0.8]
    \draw (0,0) to[V, l=$V$, invert] (0,2)
          to[R, l=$R$, i>^=$i(t)$] (3,2)
          to[L, l=$L$, v=$V_L(t)$] (3,0) -- (0,0);
\end{circuitikz}

\subsection{Resolución de la EDO}
La ecuación diferencial del circuito:
\[
V = Ri(t) + L\frac{di(t)}{dt}
\]

\subsubsection{Solución Homogénea}
\[
L\frac{di}{dt} + Ri = 0 \implies \frac{di}{i} = -\frac{R}{L}dt
\]
Integrando ambos lados:
\[
\ln i(t) = -\frac{R}{L}t + C \implies i_h(t) = Ke^{-(R/L)t}
\]

\subsubsection{Solución Particular}
Para entrada constante $V$:
\[
i_p(t) = \frac{V}{R}
\]

\subsubsection{Solución General}
\[
i(t) = i_h + i_p = Ke^{-(R/L)t} + \frac{V}{R}
\]
Con condición inicial $i(0) = 0$:
\[
0 = K + \frac{V}{R} \implies K = -\frac{V}{R}
\]
\[
\boxed{i(t) = \frac{V}{R}\left(1 - e^{-(R/L)t}\right)}
\]

\subsection{Transformada de Laplace}
Aplicando Laplace a la EDO:
\[
\mathcal{L}\{V\} = R\mathcal{L}\{i(t)\} + L\mathcal{L}\left\{\frac{di}{dt}\right\}
\]
\[
\frac{V}{s} = RI(s) + L(sI(s) - i(0))
\]
Con $i(0) = 0$:
\[
\frac{V}{s} = I(s)(R + Ls) \implies \boxed{I(s) = \frac{V}{s(R + Ls)}}
\]

\subsection{Función de Transferencia}
\[
H_{RL}(s) = \frac{I(s)}{V(s)} = \frac{1}{R + Ls} = \boxed{\frac{1/R}{1 + (L/R)s}}
\]

% ===== SECCIÓN 3: CIRCUITO RC =====
\section{Circuito RC en Serie}

\subsection{Modelado Matemático}
\begin{circuitkz}[scale=0.8]
    \draw (0,0) to[V, l=$V$, invert] (0,2)
          to[R, l=$R$, i>^=$i(t)$] (3,2)
          to[C, l=$C$, v=$V_C(t)$] (3,0) -- (0,0);
\end{circuitikz}

\subsection{Resolución de la EDO}
La ecuación diferencial:
\[
V = Ri(t) + \frac{1}{C}\int i(t) dt
\]

\subsubsection{Diferenciando}
\[
0 = R\frac{di}{dt} + \frac{i}{C} \implies \frac{di}{i} = -\frac{1}{RC}dt
\]
Solución:
\[
i(t) = Ke^{-t/(RC)}
\]

\subsubsection{Usando $V_C(t)$}
\[
i(t) = C\frac{dV_C}{dt} \implies V = RC\frac{dV_C}{dt} + V_C
\]
Solución homogénea:
\[
V_{C,h}(t) = Ke^{-t/(RC)}
\]
Solución particular:
\[
V_{C,p}(t) = V
\]
Solución general:
\[
V_C(t) = V + Ke^{-t/(RC)}
\]
Con $V_C(0) = 0$:
\[
\boxed{V_C(t) = V\left(1 - e^{-t/(RC)}\right)}
\]

\subsection{Transformada de Laplace}
Aplicando Laplace a la EDO:
\[
\frac{V}{s} = RI(s) + \frac{I(s)}{Cs}
\]
\[
I(s) = \frac{V}{s(R + 1/(Cs))} = \frac{CV}{RCs + 1}
\]
Para $V_C(s)$:
\[
V_C(s) = \frac{I(s)}{Cs} = \boxed{\frac{V}{s(RCs + 1)}}
\]

\subsection{Función de Transferencia}
\[
H_{RC}(s) = \frac{V_C(s)}{V(s)} = \boxed{\frac{1}{RCs + 1} = \frac{1}{1 + RCs}}
\]

% ===== SECCIÓN 4: COMPARACIÓN =====
\section{Tabla Comparativa}
\begin{table}[h!]
    \centering
    \begin{tabular}{lcc}
        \toprule
        \textbf{Característica} & \textbf{RL} & \textbf{RC} \\
        \midrule
        EDO & $L\frac{di}{dt} + Ri = V$ & $RC\frac{dV_C}{dt} + V_C = V$ \\
        Solución Homogénea & $Ke^{-(R/L)t}$ & $Ke^{-t/(RC)}$ \\
        Solución Particular & $\frac{V}{R}$ & $V$ \\
        Transformada de $I(s)$ & $\frac{V}{s(R + Ls)}$ & $\frac{CV}{RCs + 1}$ \\
        Función de Transferencia & $\frac{1/R}{1 + (L/R)s}$ & $\frac{1}{1 + RCs}$ \\
        Constante de Tiempo & $\tau = L/R$ & $\tau = RC$ \\
        \bottomrule
    \end{tabular}
    \caption{Comparación matemática detallada}
\end{table}

\end{document}